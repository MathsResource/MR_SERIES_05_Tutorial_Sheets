\documentclass[a4paper,12pt]{article}
%%%%%%%%%%%%%%%%%%%%%%%%%%%%%%%%%%%%%%%%%%%%%%%%%%%%%%%%%%%%%%%%%%%%%%%%%%%%%%%%%%%%%%%%%%%%%%%%%%%%%%%%%%%%%%%%%%%%%%%%%%%%%%%%%%%%%%%%%%%%%%%%%%%%%%%%%%%%%%%%%%%%%%%%%%%%%%%%%%%%%%%%%%%%%%%%%%%%%%%%%%%%%%%%%%%%%%%%%%%%%%%%%%%%%%%%%%%%%%%%%%%%%%%%%%%%
\usepackage{eurosym}
\usepackage{vmargin}
\usepackage{amsmath}
\usepackage{graphics}
\usepackage{epsfig}
\usepackage{enumerate}
\usepackage{multicol}
\usepackage{subfigure}
\usepackage{fancyhdr}
\usepackage{listings}
\usepackage{framed}
\usepackage{graphicx}
\usepackage{amsmath}
\usepackage{chngpage}
%\usepackage{bigints}

\usepackage{vmargin}
% left top textwidth textheight headheight
% headsep footheight footskip
\setmargins{2.0cm}{2.5cm}{16 cm}{22cm}{0.5cm}{0cm}{1cm}{1cm}
\renewcommand{\baselinestretch}{1.3}

\setcounter{MaxMatrixCols}{10}
\begin{document}
\begin{enumerate}
%====================================================%
 \item  A study was made of children who were hospitalized as a result of a car accident. 280 of the children were not wearing seat belts and 98 of these were seriously injured. 130 children wore seat belts and 26 were seriously injured. 
\begin{enumerate}[(a)]
\item Test the hypothesis that the rate of serious injury is the same for children who wear a seat belt or not. Clearly state your null and alternative hypotheses and your conclusion. Use a significance level of 5\%.
\end{enumerate}
\item
An exercise physiologist wants to determine if several short bouts of exercise provide the same benefit for cardiovascular fitness as one long bout of exercise. \\ 


\noindent 60 volunteers are randomly assigned to group 1 and do standardised aerobic exercise on a stationary bicycle for 30 minutes once a day, 5 days a week. 40 volunteers are randomly assigned to group 2 and do the same exercise for 10 minutes, 3 times a day, 5 days a week. Cardiovascular fitness was measured by VO2 max (maximum oxygen consumption while exercising). 
\begin{framed}
\begin{description}
	\item[Group 1] The mean change in VO2 after 12 weeks of exercise was 2.1 for group 1 with a standard deviation of 1.7.
	\item[Group 2] The mean change in VO2 after 12 weeks of exercise was 0.7 for group 2 with a standard deviation of 1. 
\end{description}
\end{framed}

\noindent Test the hypothesis that there is no significant difference between two groups are the same.

\begin{itemize}
	\item[(a)] Formally state your null and alternative hypotheses.
	\item[(b)] Compute the test statistic.
	\item[(c)] Discuss your conclusion to this test, supporting your statement with reference to appropriate values.
\end{itemize}


\newpage


\item A fruit grower wishes to test a new spray that a manufacturer claims will
reduce the amount of fruit lost due to damage by a certain insect. To test
the claim, the grower sprays 150 trees with the new spray and 120 other
trees with the standard spray. The yield of fruit was measured, in kg, for
each tree. Summary statistics were as follows. You may assume equal variance for the populations.

\begin{center}
\begin{tabular}{|c|c|c|} \hline 
                       & New spray & Standard spray\\ \hline 
Sample yield per tree  & 280       & 255\\ \hline 
Sample variance        & 650      & 640\\ \hline 
\end{tabular}
\end{center}
\begin{enumerate}[(a)]
\item Construct a 95\% confidence interval for the difference between the mean yields for the two sprays and interpret your findings. 
\item Test the hypothesis that the new spray is effective in reducing damage. Clearly state your null and alternative hypotheses and your conclusion. Use a significance level of 5\%.
\end{enumerate}

\end{enumerate}
\end{document}
