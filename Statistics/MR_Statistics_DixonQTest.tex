\documentclass[a4paper,12pt]{article}
%%%%%%%%%%%%%%%%%%%%%%%%%%%%%%%%%%%%%%%%%%%%%%%%%%%%%%%%%%%%%%%%%%%%%%%%%%%%%%%%%%%%%%%%%%%%%%%%%%%%%%%%%%%%%%%%%%%%%%%%%%%%%%%%%%%%%%%%%%%%%%%%%%%%%%%%%%%%%%%%%%%%%%%%%%%%%%%%%%%%%%%%%%%%%%%%%%%%%%%%%%%%%%%%%%%%%%%%%%%%%%%%%%%%%%%%%%%%%%%%%%%%%%%%%%%%
\usepackage{eurosym}
\usepackage{vmargin}
\usepackage{amsmath}
\usepackage{graphics}
\usepackage{epsfig}
\usepackage{enumerate}
\usepackage{multicol}
\usepackage{subfigure}
\usepackage{fancyhdr}
\usepackage{listings}
\usepackage{framed}
\usepackage{graphicx}
\usepackage{amsmath}
\usepackage{chngpage}
%\usepackage{bigints}

\usepackage{vmargin}
% left top textwidth textheight headheight
% headsep footheight footskip
\setmargins{2.0cm}{2.5cm}{16 cm}{22cm}{0.5cm}{0cm}{1cm}{1cm}
\renewcommand{\baselinestretch}{1.3}

\setcounter{MaxMatrixCols}{10}

\begin{document}
\begin{enumerate}

% Part A - Dixon Q Test (5 Marks)
\item Use the Dixon Q-test to determine if there is an outlier present in this sample data. You may assume
a significance level of 5\%.
\[ 131, 139, 107, 117, 123, 127, 122, 132, 135\]
\begin{enumerate}[(a)]
\item  State the null and alternative hypotheses for this test.
\item  Compute the test statistic?
\item  State the appropriate critical value.
\item  What is your conclusion to this procedure?
\end{enumerate}


% Part A - Dixon Q Test (5 Marks)
\item Use the Dixon Q-test to determine if there is an outlier present in this sample data. You may assume
a significance level of 5\%.
\[ 131, 136, 103, 117, 123, 127, 122, 132, 135\]
\begin{enumerate}[(a)]
\item  State the null and alternative hypotheses for this test.
\item  Compute the test statistic?
\item  State the appropriate critical value.
\item  What is your conclusion to this procedure?
\end{enumerate}




% Part A - Dixon Q Test (5 Marks)
\item Use the Dixon Q-test to determine if there is an outlier present in this sample data. You may assume
a significance level of 5\%.
\[ 133, 139, 159, 129, 123, 
  137, 142, 124, 132, 136\]
\begin{enumerate}[(i)]
\item  State the null and alternative hypotheses for this test.
\item  Compute the test statistic?
\item  State the appropriate critical value.
\item  What is your conclusion to this procedure?
\end{enumerate}
\end{enumerate}

\subsection*{Critical Values for Dixon Q Test}
{
	\Large
	\begin{center}
	\begin{tabular}{|c|c|c|c|}
		\hline  n  & $\alpha=0.10$  & $\alpha=0.05$  & $\alpha=0.01$  \\ \hline
		3  & 0.941 & 0.970  & 0.994 \\ \hline
		4  & 0.765 & 0.829 & 0.926 \\ \hline
		5  & 0.642 & 0.710  & 0.821 \\ \hline
		6  & 0.56  & 0.625 & 0.740  \\ \hline
		7  & 0.507 & 0.568 & 0.680  \\ \hline
		8  & 0.468 & 0.526 & 0.634 \\ \hline
		9  & 0.437 & 0.493 & 0.598 \\ \hline
		10 & 0.412 & 0.466 & 0.568 \\ \hline
		11 & 0.392 & 0.444 & 0.542 \\ \hline
		12 & 0.376 & 0.426 & 0.522 \\ \hline
		13 & 0.361 & 0.401  & 0.503 \\ \hline
		14 & 0.349 & 0.396 & 0.488 \\ \hline
		15 & 0.338 & 0.384 & 0.475 \\ \hline
		16 & 0.329 & 0.374 & 0.463 \\ \hline
	\end{tabular} 
\end{center}
}

\end{document}
