

%% -- https://stattrek.com/probability-distributions/negative-binomial.aspx

\documentclass[a4paper,12pt]{article}
%%%%%%%%%%%%%%%%%%%%%%%%%%%%%%%%%%%%%%%%%%%%%%%%%%%%%%%%%%%%%%%%%%%%%%%%%%%%%%%%%%%%%%%%%%%%%%%%%%%%%%%%%%%%%%%%%%%%%%%%%%%%%%%%%%%%%%%%%%%%%%%%%%%%%%%%%%%%%%%%%%%%%%%%%%%%%%%%%%%%%%%%%%%%%%%%%%%%%%%%%%%%%%%%%%%%%%%%%%%%%%%%%%%%%%%%%%%%%%%%%%%%%%%%%%%%
\usepackage{eurosym}
\usepackage{vmargin}
\usepackage{amsmath}
\usepackage{graphics}
\usepackage{epsfig}
\usepackage{enumerate}
\usepackage{multicol}
\usepackage{subfigure}
\usepackage{fancyhdr}
\usepackage{listings}
\usepackage{framed}
\usepackage{graphicx}
\usepackage{amssymb}
\usepackage{chngpage}
%\usepackage{bigints}

\usepackage{vmargin}
% left top textwidth textheight headheight
% headsep footheight footskip
\setmargins{2.0cm}{2.5cm}{16 cm}{22cm}{0.5cm}{0cm}{1cm}{1cm}
\renewcommand{\baselinestretch}{1.3}

\setcounter{MaxMatrixCols}{10}

\begin{document}
\large


\noindent Alex is a high school basketball player. He is a 70\% free throw shooter. That means his probability of making a free throw is 0.70. 

\begin{enumerate}[(a)]
    \item During the season, what is the probability that Alex makes his third free throw on his fifth shot?
    \item  What is the probability that Alex makes his first free throw on his fifth shot?
\end{enumerate}

%%%%%%%%%%%%%%%%%%%%%

\subsection*{Solution}
\subsection*{Part (a)}
This is an example of a negative binomial experiment. The probability of success ($p$) is 0.70, the number of trials ($k$) is 5, and the number of successes ($r$) is 3.

\noindent To solve this problem, we enter these values into the negative binomial probability mass function.

\begin{framed}
\noindent \textbf{Probability Mass Function}

\noindent Suppose a negative binomial experiment consists of $k$ trials and results in $r$ successes. If the probability of success on an individual trial is $p$, then the negative binomial probability is:
\[
{\displaystyle f(k;r,p)\equiv \Pr(X=k)={k-1 \choose r-1}\cdot p^{r}(1-p)^{k-r},}\]

\noindent N.B. There are other formulations of the PMF for different scenarios.\\

\end{framed}


\begin{eqnarray*}
P(X=k)  &=& {k-1 \choose r-1} \times p^r \times (i-p)^{k - r}\\
 &=& {4 \choose 2} \times 0.7^3 \times  0.3^2 \\
  &=& 6 \times 0.343 \times  0.09 \\
&=& 0.18522 \\
\end{eqnarray*}

\noindent Thus, the probability that Alex will make his third successful free throw on his fifth shot is 0.18522.

%%%%%%%%%%%%%%%%%%%%%%%%
\newpage
\subsection*{Part (b)}
 What is the probability that Alex makes his first free throw on his fifth shot?\\
 \medskip

\noindent This is an example of a geometric distribution, which is a special case of a negative binomial distribution. Therefore, this problem can be solved using the negative binomial formula or the geometric formula. We demonstrate both approaches.


\subsection*{Negative Binomial Approach}

The probability of success ($p$) is 0.70, the number of trials ($k$) is 5, and the number of successes ($r$) is 1. 

\begin{eqnarray*}
P(X=k)  &=&  {k-1 \choose r-1} \times p^r \times (1-p)^{k - r} \\
 &=&  {4 \choose 0} \times 0.7^1 \times 0.3^{4}\\
 &=&  0.00567\\
\end{eqnarray*}

\newpage 
\subsection*{Geometric Approach}
We will also demonstrate a solution based on the geometric formula.\\ 
\noindent (N.B This is the Type I geometric distribution).
\begin{eqnarray*}
P(X=k)  &=&   p \times (1-p)^{k - 1}\\
 &=&  0.7 \times 0.3^4 \\
  &=&  0.00567\\
\end{eqnarray*}
Notice that each approach yields the same answer.

\end{document}