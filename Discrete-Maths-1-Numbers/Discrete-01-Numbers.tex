Convert the following statements into symbols:


\begin{itemize}
\item $\sqrt{2}$ is less than 1.5 and greater than 1.4
\item $\sqrt{2}$ is greater than or equal to 5
\end{itemize}



Say which of the set the following numbers belong to.

If they belong to more than one of these sets, give all the sets.

$\sqrt{2}$
$\frac{3}{7}$
% %------------------------------------------------------
\section*{Part 1. Number Systems}


%----------------------------------------- %
\section{Inequality Operators}
%------------------------------------------------------------------------- %

%\frametitle{Inequality Symbols}

Given $x = \sqrt{2}$ determine whether the following statements are true or false:

\begin{itemize}
\item[(i)] $x \leq 2$
\item[(ii)] $1.42 > x > 1.41$
\item[(iii)] x is a rational number
\item[(iv)] $\sqrt{2} = 2$
\end{itemize}


\subsection*{Part D : Real and Rational Numbers}
\begin{itemize}
\item[(i)] Express the recurring decimal $0.727272\ldots$ as a rational number in its simplest form.
\end{itemize}
%-------------------------------------------------------- %
\begin{itemize}
\item[(i)] Given x is the irrational positive number $\sqrt{2}$, express $x^8$ in binary notation.
\item[(ii)] From part (i), is $x^8$ a rational number?
\end{itemize}







\item Perform the following binary multiplications.
%\begin{multicols}{2}
%\begin{itemize}
%\item[a)] $(1001000)_{2} \div ( 1000)_{2}$
%\item[b)] $(101101)_{2} \div (1001)_{2}$
%\item[c)] $(1001011000)_{2} \div (101000)_{2}$
%\item[d)] $(1100000)_{2} \div (10000)_{2}$
%\end{itemize}
%\end{multicols}

%----------------------------------------------------------------%

\item Perform the following binary divisions.
%\begin{multicols}{2}
%\begin{itemize}
%\item[a)] $(1001000)_{2} \div ( 1000)_{2}$
%\item[b)] $(101101)_{2} \div (1001)_{2}$
%\item[c)] $(1001011000)_{2} \div (101000)_{2}$
%\item[d)] $(1100000)_{2} \div (10000)_{2}$
%\end{itemize}
%\end{multicols}

\begin{enumerate}
\item Which of the following binary numbers is the result of this binary division: $(111001)_{2} \div ( 10011)_{2}$. % (57) /  (19)
\begin{multicols}{2}
\begin{itemize}
\item[a)] $(10)_2$ %2
\item[b)] $(11)_{2}$ %3
\item[c)] $(100)_{2}$ %4
\item[d)] $(101)_{2}$ %5
\end{itemize}
\end{multicols}
\item Which of the following binary numbers is the result of this binary division: $(101010)_{2} \div ( 111 )_{2}$. % (42) /  (7)
\begin{multicols}{2}
\begin{itemize}
\item[a)] $(11)_2$ %3
\item[b)] $(100)_{2}$ %4
\item[c)] $(101)_{2}$ %5
\item[d)] $(110)_{2}$ %6
\end{itemize}
\end{multicols}
\item Which of the following binary numbers is the result of this binary division: $(1001110)_{2} \div ( 1101 )_{2}$. % (78) /  (13)
\begin{multicols}{2}
\begin{itemize}

\item[a)] $(100)_{2}$ %4
\item[b)] $(110)_{2}$ %6
\item[c)] $(111)_{2}$ %7
\item[d)] $(1001)_2$ %9
\end{itemize}
\end{multicols}
\end{enumerate}


\end{enumerate}

% \newpage
\section*{Part D: Natural, Rational and Real Numbers}
\begin{framed}
\begin{itemize}
\item $\mathbb{N}$ : natural numbers (or positive integers) $\{1,2,3,\ldots\}$
\item $\mathbb{Z}$ : integers $\{-3,-2,-1,0,1,2,3,\ldots\}$
\begin{itemize}
\item[$\ast$] (The letter $\mathbb{Z}$ comes from the word \emph{Zahlen} which means ``numbers" in German.)
\end{itemize}
\item $\mathbb{Q}$ : rational numbers
\item $\mathbb{R}$ : real numbers
\item $\mathbb{N} \subseteq \mathbb{Z } \subseteq \mathbb{Q} \subseteq \mathbb{R}$
\begin{itemize}
\item[$\ast$] (All natural numbers are integers. All integers are rational numbers. All rational numbers are real numbers.)
\end{itemize}
\end{itemize}
