Matrices
1. Scalar Multiplication
2. Scalar Addition
3. Order of multiplication
4. Identity matrix (number of elements)
5. tranpose of an idenity matrix
6. transpose of a matrix
7. Dimensions of a matrix
8. Compatability of matrix dimensions
9. Gaussian elimination augmented matrix theory question
10. Row Echelon Form
11. 


Part 5: Set Theory
=============================
### Introduction to Set Theory

Set theory is concerned with the concept of a set, essentially a collection of objects that we call elements. 
Because of its generality, set theory forms the foundation of nearly every other part of mathematics

### What is a Set?

A mathematical set is defined as an unordered collection of distinct elements. That is, elements of a set can be listed in any order and elements occurring more than once are equivalent to occurring only once.
 
We say that an element is a member of a set. An element of a set can be anything. It's easiest to begin with only numbers as elements. For that reason, most of the examples in this book will only include numbers, but this is only a technique to make the topic less abstract.

### set operations

Based on the preceding definitions, we can derive some useful properties for the operations on sets. The proofs of these properties are left as an exercise to the reader.
 
The union and intersection operations are ***symmetric***. 



####Quotients and Remainders

A "quotient" is the result of a division problem. 

For example: For 6/3, "2" is the quotient, 6 is the dividend, and 3 is the divisor. 

The remainder is the number "left over" in a division problem when you're only interested in whole numbers 
(i.e. no decimals or fractions in the answer). 

For example, in 5/2, the quotient is "2" and the remainder is "1." 

If you take some of the more advanced math courses, you'll also see the remainder referred to as the "modulus," 
and coming in useful in many applications



###Functions
A function is a relationship between two sets of numbers. 
We may think of this as a mapping; a function maps a number in one set to a number in another set. Notice that a function maps values to one and only one value. Two values in one set could map to one value, 
but one value must never map to two values: that would be a relation, not a function.

One important kind of relation is the function. A function is a relation that has exactly one output for every possible input in the domain. (The domain does not necessarily have to include all possible objects of a given type. In fact, we sometimes intentionally use a restricted domain in order to satisfy some desirable property.) 

The relations discussed above (flavors of fruits and fruits of a given flavor) are not functions: the first has two possible outputs for the input "apples" (sweetness and tartness); and the second has two outputs for both "sweetness" (apples and bananas) and "tartness" (apples and oranges).




Part 2 : Numbers
==================================
*** Elements of Section 1 of Main Module ***

### Prime Numbers
Prime numbers are the building blocks of the integers. A prime number is a positive integer greater than one that has only two divisors: 1, and the number itself. For example, 17 is prime because the only positive integers that divide evenly into it are 1 and 17. The number 6 is not a prime since more than two divsors 1, 2, 3, 6 divide 6. Also, note that 1 is not a prime since 1 has only one divisor.
 
#### Some prime numbers
 
The prime numbers as a sequence begin
<pre><code>
 2, 3, 5, 7, 11, 13, 17, 19, 23, ...
</code></pre>

### Powers, Logarithms and Exponentials

### Scientific and Floating Point Notation




office chair
rug
