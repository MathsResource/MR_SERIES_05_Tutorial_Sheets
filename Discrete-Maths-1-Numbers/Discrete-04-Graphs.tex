\documentclass[]{article}
\voffset=-1.5cm
\oddsidemargin=0.0cm
\textwidth = 470pt
\usepackage[utf8]{inputenc}
\usepackage[english]{babel}
\usepackage{framed}

\usepackage{multicol}
\usepackage{amsmath}
\usepackage{amssymb}
\usepackage{enumerate}
\usepackage{multicol}

% Some CPSC 259 Sample Exam Questions on Graph Theory (Part 6) Sample Solutions 

% DON’T LOOK AT THESE SOLUTIONS UNTIL YOU’VE MADE AN HONEST ATTEMPT AT ANSWERING THE QUESTIONS YOURSELF. 
\begin{documet}
1.
{3 marks}  

Can a simple graph have 5 vertices and 12 edges?  If so, draw it;  if not, explain why it is not possible to have such a graph. 

ANSWER: In a simple graph, no pair of vertices can have more than one edge between them.  
In other words, there are no parallel edges. 

For a simple graph, the “densest” graph we can get is one in which every vertex is connected to every other vertex.  
This is called a complete graph.  
The maximum number of edges in the complete graph containing 5 vertices is given by K5:  which is C(5, 2) edges = “5 choose 2” edges = 10 edges.  
Since 12 > 10, it is not possible to have a simple graph with more than 10 edges. 2.
{6 marks}  


Suppose that in a group of 5 people: A, B, C, D, and E, the following pairs of people are acquainted with each other.
\item A and C
\item A and D
\item B and C
\item C and D
\item C and E 

a)Draw a graph G to represent this situation. 

b)List the vertex set, and the edge set, using set notation.  
In other words, show sets V and E for the vertices and edges, respectively, in G = {V, E}. 
c)Draw an adjacency matrix for G. 

ANSWER: a)One such graph for G is:   A B    C D    E 
b)For sets V and E, any order to the elements is fine.  

Furthermore, in edge set E, you can specify (A, C) or (C, A);  they mean the same thing. 
V = {A, B, C, D, E} 
E = {(A, C), (A, D), (B, C), (C, D), (C, E)} c)

Adjacency matrix (0 = no edge;  1 = edge):         
A     B     C     D     E     
A     0     0     1     1     0     
B     0     0     1     0     0     
C     1     1     0     1     1     
D     1     0     1     0     0     
E     0     0     1     0     0     

3.
{3 marks}  
How many more edges are there in the complete graph K7 than in the complete graph K5? 
ANSWER: C(7, 2) – C(5, 2) = 21 – 10 = 11 



4.
{4 marks}  

Given a graph for a tree (with no designated root), briefly describe how a root can be chosen so that the tree has maximum height.  
Similarly, describe how a root can be chosen so that the tree has minimum height.  
(Note that path length is described as the number of edges that need to be traversed between two vertices.) 
ANSWER: For the maximum height, choose either end of the longest path as the root.  
For the minimum height, choose the vertex at the half-way point of the path. 5.


Perform  a  breadth-first  search  of  the  following  graph,  where  E  is  the  starting  node.  In other words, show the output if we issue the call BFS(E).  
Provide two cases:  (a) Use a counterclockwise ordering from the top (12 o’clock position);  and (b) Use a clockwise ordering from the top. 

https://exams.ubccsss.org/potential/sample_exam_questions_6_soln.pdf




5.5 Exercises 5 
\newpage
\begin{enumerate}
    \item 

1. For the graph shown below, find: 
(a) all the edges incident with vi; 
(b) all the vertices adjacent to v3; 
(c) all the edges adjacent to e2; 
(d) all the loops; 
(e) the number of its connected components; 
(f) deg(v4), deg(v6) and deg(v8); 
(g) the degree sequence of the graph; 
(h) all the cycles of lengths 2, 3 and 4 respectively. 

C4 
V7 
C9 
\item 
 In each of the following cases, either construct a graph with the specified properties or say 
why it is not possible to do so. 
\begin{enumerate}[(a)]
\item  A graph with degree sequence 3, 3, 2,1. 
\item A simple graph with degree sequence 3, 3, 2, 1, 1. 
\item A simple graph with degree sequence 4, 3, 2, 1. 
\item A simple 3-regular graph with 6 vertices. 
\end{enumerate}
\item  Say why every graph has an even number of vertices of odd degree. 
\item Suppose that eight sites are connected in a network. The number of other sites to which each 
site has a direct connection is given by the following sequence. 
\[5,3,2,7,1,2,6,4.\] 
\begin{enumerate}[(a)]
\item Describe how a communications network such as this can be modelled by a graph, saying 
what the vertices represent and when two vertices are adjacent. 
\item  What information about the graph is given by the sequence of numbers above? 
\item Find how many pairs of sites have a. direct connection between them, giving a brief 
explanation of your method. 
\item  Say why it is impossible to construct a network with 9 sites, in which each site has a 
direct connection to exactly 5 of the other sites. 
\end{enumerate}
%%-- CHAPTER 5. INTRODUCTION TO GRAPH THEORY 79 



\item  Arc the graphs G and H shown below isomorphic? If you think they are isomorphic, label each vertex of H with the same letter as the corresponding vertex in G. Otherwise, give a reason why the two graphs are not isomorphic. 

\item  Draw two simple non-isomorphic graphs with degree sequence 3, 2, 2,1,1, 1. Give a reason why the graphs you have drawn are not isomorphic. 

\item Construct an adjacency matrix A(G) for the graph G shown below. 

\begin{enumerate}[(a)]
    \item  What inform ation does the sum of the elements in any row of A(G) give you about the graph G? 
    \item What information does the sum of all the elements in the matrix tell you about G? 
    \item Would these rules hold in the case of a graph with loops? 
\end{enumerate}
\item  Let G be a simple graph with vertex set V (G) = {vi, v2, v3, v4, v5} and adjacency lists as follows: 
VI • V2 V3 V4 V2 Vl v3 v4 vs 
vi v.1 v4 vi v2 v3. v5 v2 
\begin{enumerate}[(a)]
\item List the degree sequence of G. 
\item Draw the graph of G. 
\item Find two distinct paths of length 3, starting at va and ending at v4. 
\item Find a 4 cycle in G. 
\end{enumerate}

\item Let Ky, be the simple graph with vertices vZ, v2, v3, ..., i.'3 in which each vertex is joined to every other vertex by an edge. 
\begin{enumerate}[(a)]
\item Draw K6. 
\item Determine the number of edges of K6. 
\item Determine the number of paths from vi to v2 of length two.
\item Find an expression in terms of n for the number of paths from vi to v2 of length two in kn.  
\end{enumerate}


\item Write out the degree sequence of the following graph.
%\begin{figure}[h!]
%\centering
%\includegraphics[width=0.5\linewidth]{./graph2.jpg}
%%\caption{}
%\label{fig:graph2}
%\end{figure}
\item State the vertices that comprise a cycle of length 5 in both of the following graphs.
%\begin{figure}[h!]
%\centering
%\includegraphics[width=0.7\linewidth]{./graph20.jpg}
%
%\label{fig:graph20}
%\end{figure}
\end{enumerate}





\end{document}


%%--https://www.tutorialandexample.com/graph-%%--http://www.maths.usyd.edu.au/u/billp/MATH2069/information.htmltheory-trees-in-graph/

\subsection{Question 10}

(a) Given the following adjacency matrices A and B where
A =

1 0 1
0 1 2
1 2 0

,B =

1 2 0
2 0 1
0 1 1

%MAKE NO

%--------------------------------------------%

(i) Say whether or not the graphs they represent are isomorphic.
(ii) Calculate A2 and A4 and say what information each gives about the graph
corresponding to A. [6]



%%%--------------------%% %%%--------------------%%

%%%--------------------%% %%%--------------------%%

%%%--------------------%% %%%--------------------%% %%%--------------------%%

%%%--------------------%% %%%--------------------%% %%%------------------------%

