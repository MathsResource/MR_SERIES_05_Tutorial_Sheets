
\begin{document}{Relations}

Let $S = \{\{1\}, \{1, 2\}, \{1, 2, 3\}, \{1, 2, 4\}\}$. Define a relation $\mathcal{R}$
between the elements of $S$ by


\begin{center}
$X$ is related to $Y$ if $X \subset Y$.
\end{center}


\subsection{2001}
Question 8 Let S = (a, b, c, d) and suppose that a relation R. is defined on S in precisely the following cases: 
aRa, aRb, aRc, bRb, bR.c, cR.d. 
(a) Draw the relational digraph for 7?. on S. [2) (b) The relation R is not reflexive. Which minimal set of pairs should be added to R to make it reflexive? [2] (c) The relation R is not symmetric. Which minimal set of pairs should be added to R to make it symmetric? [2] (d) The relation R is not transitive. Which minimal set of pairs should be added to R to make it transitive? [2] (e) Is the relation R anti-symmetric? Justify your answer. (2] 
\subsection{2002}

Question 7 (a) Let S be a set and let TZ be a relation on S. Explain whe. .• means to say that Te is: 
(i) reflexive; (ii) symmetric: (iii) antisymmetric; (iv) transitive. 
(b) Let S (1,2,3,4.6,12). Define a relation 1Z between the elements of S by -*x is related toy whenever x divides (i) Draw the relationship digraph, '2' (ii) Determine whether or not TZ, is reflexive, symmetric. antisymmetric , :• transitive. For the case(s) when one of these properties does not 11( justify your answer by giving an example to show that the property d: not hold. 
OW State. giving reasons. whether or not Te is an equivalence relation Z.L whether or not it is a partial order. 




\subsection{2003}
Question 6 (a) Let S be the set {2, 3, 4, 5,6,7} and a relation R. is defined be-
tween the elements of AS by 
xisrelatedtayif x--yE {O, 2, 4}". 
(I) Draw the relationship digraph. 
(ii) Determine whether or not R. is reflexive, symmetric or transitive. In cases 
where one of these properties does not hold give an example to show that 
it does not hold. 
(iii) State, with reason, whether Te is a partial order or not. 
[8] 
(b) Let X be the set of all 3 bit binary strings and Y be the set {O, 1, 2, 31. The 
relationship R is defined to be the subset of X x Y where the elements x and y 
are related, (x14), if y is equal to the number of zeros in x. List the elements 
of 7L. [2] 

\subsection{2004} 
Question 9 Let S be the set fa, b, c, d}. 
(a) (i) Describe briefly how each subset of S can be represented by a unique 4-bit 
binary string. 
(ii) Write down the string corresponding to the subset {a, c, d} and the subset 
corresponding to the string 0110. 
(iii) What is the total number of subsets of 5? [4] 
(b) R is a relation defined on S in precisely the following cases: 
04; bRe; cRb; CRC; cRdi dRa-(1) Draw the relationship digraph for R on S. 
\begin{enumerate}[(i)]
\item  The relation R is not reflexive. Which minimal set of pairs should be 
added to R to make it relexive? 
\item  The relation R is not symmetric. Which minimal set of pairs should be 
added to R to make it symmetric? 
\item  The relation fi is not transitive. Which minimal set of pairs should be 
added to R to make it transitive? 
\item Is the relation R anti-symmetric? Justify your answer. 
\end{enumerate}


- c) ca-swe-414 4 eta-s-vA.-t rf it) dr': "%r<41 v ect:3t1 6 ot z.t sliKt= I ca. e) C h0 oe Sx "if - 2-0. rito Re., rut) cita, a-A. a 124 it)) Loka *-4 ele, ir) 1+ co A Or 41.'.e &jilt., or 1,241.x._ 2. . _ 

\subsection{2005}
question 7 (a) In a tournament with four players A, B, C, D every player plays every other player exactly once. The results are as follows: 
AbeatsBandC BbeatsC DbeatsAand 
(I) Draw a digraph to model this information. Explain what the vertices of the digraph represent and what an arc from one vertex to another represents. 
(ii) R. is the relation represented by this digraph. Determine whether or not R. is reflexive, symmetric or transitive. Give an example to support your answer in each case where the property does not hold and justify your answer in each case where the property does hold. [6] 
(b) Let S = (a, b, c}. For each of the following statements concerning relations, say whether or not it is true for any relation R. on S. If a statement is false give a reason for your answer. 
(i) /Z is reflexive if .7?.a and bRb. (ii) If R. is symmetric then .74 and bRa. (iii) If 7Z is anti-symmetric then it is not symmetric. (iv) If 7Z is not an equivalence relation then it is a partial order. 
) ,414 cgo re4-1-6-1 .;1Cticaki, .0, /44.0 

\subsection{2006}
L pwr h".3-ple„91Li. 
AA C L74. C ott-. xizz 
fi3 cvt2u, cdzc " htz‘k rtwa-sb-44 el‘ j i fit Tc) rvux-9 Fo.e la() it,. .e G- Q-0 
V 


.1.r a4 J• 
ittii-e-11 L-1 :--7 


question 8 
8. 
(a) Consider a set S = (0,1,2,3, 4,51. R1 is the relation such that xRiy, if 
x — y = 2 and R2 is the relation such that xR2y if x — y is even, for all x and 
y 
E S. 
(I) illustrate the relations R1 and R2,using a separate digraph for each. 
\begin{enumerate}[(i)]
\item Complete the following table: 
Reflexive Symmetric Anti-symmetric `Ransitive 
\item  One of these relations is an equivalence relation. Say which relation this 
is and give the partition on S created by this relation. 
\end{enumerate}
(b) Another relation is defined on a population of people such that x is related to 
y if x is a brother of y, for all x and y in the population. Say whether or not 
this relation is reflexive, symmetric or transitive, explaining briefly what this 
means in terms of the relation in each case. (Note: in this instance "brother 
of means x and y have the same parents and are both male.) Fij 
(a) (1) See page 2 of solutions 
(b) iii) R.2 is an equivalence relation with equivalence classes {O,2,4} and 
{1,3,5}. The partition is {pi ,11j1. 
reflexive symmetric anti symetric transitive 
(ii) 
R2 x 44( 
(c) Ft is not relexive since no-one is a brother of themselves 
R is not symmetric since x may be a brother of y but y is not always 
a brother of x. 
R. is transitive since, if x is a brother of y and y is a brother of z then 
x is also a brother of z for all x,y and z in the population. 


(c) Another relation, R2, is defined between the chickens on Home Farm. Let x 
and y be chickens on Home Farm, then 
xR2yif andonlyi f xandvhavethesamemother 
The mothers of the chickens on Home Farm are either Flora or Harriet from a 
neighbouring farm. Harriet is the mother of Amy, Daisy and Eve. Flora is the 
mother of Beth and Carol. 
Justifying your answer, say whether R2 is an equivalence relation on the set 
of chickens at Home Farm. If this is an equivalence relation write down the 
equivalence classes. [4] 
\end{document}