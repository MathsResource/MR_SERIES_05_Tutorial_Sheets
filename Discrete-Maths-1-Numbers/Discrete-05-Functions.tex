
\begin{enumerate}


\item 

What conditions must be satisfied for a function to have an inverse.
\begin{enumerate}
\item One-one and onto
\item One-to-one only
\item onto only
\item Neither onto nor One-to-One
\end{enumerate}
\item 

If f is a function for which the rule is f(x) = 7/8  - x, where x is real, the rule for the inverse function f-1 is:
A	f -1(x) = 8/7 + x
B	f -1(x) = -8/x + 7 
C	f -1(x) = 2x + 73/4 
D	f -1(x) = 7/8 - x  % (This one)
E	f -1(x) = 8/7 + x

\item 
Which of the following functions is not one-to-one?
\begin{enumerate}
\item 	$f(x) = 9 - x^2, x \geq 0$
\item 	$f(x) = 1/x^2  - 9$ %  (This one)
\item	f(x) = 1 -9x
\item	$f(x) = \sqrt{x}$
\item	$f(x) = 3/x $
\end{enumerate}
\item 
The range of the function with rule f(x) = |x - 4| + 3 is:
A	(4, \infty)
B	R
C	[3, \infty) This one
D	(4, \infty)
E	(-1, \infty)

% http://www.analyzemath.com/college_algebra/problems_5.html
\item Let X = {1, 2, 3, 4} and Y={A,B,C,D}. Determine whether each relation is a function, with X -> Y.
(a) f = {(2, C), (1, D), (2, A), (3,B), (4, D)}

No:  Two different ordered pairs (2, C) and (2, A) in f have the same number 2 as their first coordinate

\item Let $X = {1, 2, 3, 4}$. Determine whether the following relation on X is a function.

(b) g = {(3, A), (4, B), (1, C)}

No The element 2 does not appear as the first coordinate in any ordered pair in g.
\end{enumerate}


\subsection{Functions}

%--------------------------------------------%

(i) Say whether or not the graphs they represent are isomorphic.
(ii) Calculate A2 and A4 and say what information each gives about the graph
corresponding to A. [6]
(b) (i) Write down the augmented matrix for the following system of equations.

\[2x + y - z = 2\]
\[x - y + z = 4\]
\[x + 2y + 2z = 10\]
(ii) Use Gaussian elimination to solve the system. [4]


\begin{enumerate}


\item Consider the floor function $f : R \rightarrow Z$ given the rule

\[ f(x) = \lfloor \frac{x+1}{2} \rfloor \]

\begin{enumerate}[(i)]
\item evaluate $f(6)$ and $f(-6)$
\item Show that f(x) is not one-to-one
\end{enumerate}


\end{enumerate}

\end{document}
\end{document}
%=====================================================================%
\subsection*{Question 4}
Let $S$ be the set of all 4 bit binary strings. The function $f : S \rightarrow Z$
is defined by the rule:
\[f(x) = \mbox{ the number of zeros in x} \] for each binary string $x \in S$.
Find:
\begin{itemize}
\item[(a)]  Answer the following questions
\begin{itemize}
\item[(i)] the number of elements in the domain
\item[(ii)] f(1010)
\item[(iii)] the set of pre-images of 1
\item[(iv)] the range of f. 
\end{itemize}
\item[(b)]  Decide whether the function $f$, as defined above, has either the one to one or
the onto property, justifying your answers. 
\item[(c)]  State the condition to be satisfied by a function $f : X \rightarrow Y$ for it to have an
inverse function $f^{-1} : Y \rightarrow X$.
\item[(d)]  Define the inverse functions for each of the following:
\end{itemize}

%=====================================================================%%------------------------------------%

\section*{Question 4}

A function f: X-> Y , where $X = \{p,q,r,s\}$ and $Y =\{1,2,3,4,5\}$
is given by the subset of $X \times Y$


\begin{itemize}
\item Show f as an arrow diagram
\item state the domain, the co-domain, and the range of f
\item Say why f does not have the one-to-one property and why f does 
not have the "onto" property, giving a specific counter example in each case.
\end{itemize}