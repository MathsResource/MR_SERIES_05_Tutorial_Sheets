

\usepackage{vmargin}
% left top textwidth textheight headheight
% headsep footheight footskip
\setmargins{2.0cm}{2.5cm}{16 cm}{22cm}{0.5cm}{0cm}{1cm}{1cm}
\renewcommand{\baselinestretch}{1.3}

\setcounter{MaxMatrixCols}{10}

\begin{document}
\large 


%%- http://webpages.iust.ac.ir/matashbar/teaching/schaum_probability.pdf

%%-- PAge 287
%%-- 9.4



Derive Eq. (9.15). 

%%%%%%%%%%%%%%%%%%%%%%%%%%%%%%%%%%%%%
 The ratio ${\rho = \frac{\lambda}{\mu}}$ is sometimes referred to as the trafJic intensity of the 

system. The traffic intensity of the system is defined as 
mean service time mean arrival rate Traffic intensity = - mean interarrival time mean service rate 
The average number of customers in the system is given by (Prob. 9.4) 
The

\newpage 

%%%%%%%%%%%%%%%%%%%%%%%%%%%%%%%%%%%%%%
Since $p_n$ is the steady-state probability that the system contains exactly $n$ customers, using Eq. (9.14), 
the average number of customers in the M/M/1 queueing system is given by \[ L = \sum^{\infty}_{n=0}p_{n} =\sum^{\infty}_{n=0} n(1-\rho) \rho^{n} = (1-\rho) p_{0}\left(\sum^{\infty}_{n=0}np^{n}\right)  \]

where ${\rho = \frac{\lambda}{\mu} < 1}$ . Using the algebraic identity 
\[\sum^{\infty}_{n=0}nX^{n} = \frac{x}{(1-x)^3} \mbox{ where } \mid x \mid < 1 \]

we obtain 

\[L =  \frac{\rho}{(1-\rho)} = \frac{\frac{\lambda}{\mu}}{1- \frac{\lambda}{\mu} } = \frac{\lambda}{\mu-\lambda}\]
\end{document}

Let's derive Equation (9.15) step by step. 

Given that the traffic intensity \(\rho = \frac{\lambda}{\mu}\), and \(p_n\) is the steady-state probability that the system contains exactly \(n\) customers, we use Equation (9.14). The average number of customers in the M/M/1 queueing system is given by:

\[ L = \sum^{\infty}_{n=0} n p_{n} \]

We know that the probability \(p_n\) is given by:

\[ p_{n} = (1 - \rho) \rho^{n} \]

So, substituting this into our equation, we get:

\[ L = \sum^{\infty}_{n=0} n (1 - \rho) \rho^{n} \]

Factor out \((1 - \rho)\):

\[ L = (1 - \rho) \sum^{\infty}_{n=0} n \rho^{n} \]

To evaluate the summation, we use the algebraic identity:

\[ \sum^{\infty}_{n=0} n x^{n} = \frac{x}{(1 - x)^{2}}, \quad \text{where} \ |x| < 1 \]

By substituting \(\rho\) for \(x\), we get:

\[ \sum^{\infty}_{n=0} n \rho^{n} = \frac{\rho}{(1 - \rho)^{2}} \]

So:

\[ L = (1 - \rho) \cdot \frac{\rho}{(1 - \rho)^{2}} \]

Simplify this expression:

\[ L = \frac{\rho}{1 - \rho} \]

Finally, substituting \(\rho = \frac{\lambda}{\mu}\):

\[ L = \frac{\frac{\lambda}{\mu}}{1 - \frac{\lambda}{\mu}} = \frac{\lambda}{\mu - \lambda} \]

Thus, we have derived Equation (9.15):

\[ L = \frac{\lambda}{\mu - \lambda} \]

I hope that clarifies the derivation for you! If you have any more questions or need further explanation, feel free to ask.