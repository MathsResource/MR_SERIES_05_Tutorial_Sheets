

\documentclass[a4paper12pt]{article}
%%%%%%%%%%%%%%%%%%%%%%%%%%%%%%%%%%%%%%%%%%%%%%%%%%%%%%%%%%%%%%%%%%%%%%%%%%%%%%%%%%%%%%%%%%%%%%%%%%%%%%%%%%%%%%%%%%%%%%%%%%%%%%%%%%%%%%%%%%%%%%%%%%%%%%%%%%%%%%%%%%%%%%%%%%%%%%%%%%%%%%%%%%%%%%%%%%%%%%%%%%%%%%%%%%%%%%%%%%%%%%%%%%%%%%%%%%%%%%%%%%%%%%%%%%%%
\usepackage{eurosym}
\usepackage{vmargin}
\usepackage{amsmath}
\usepackage{graphics}
\usepackage{epsfig}
\usepackage{enumerate}
\usepackage{multicol}
\usepackage{subfigure}
\usepackage{fancyhdr}
\usepackage{listings}
\usepackage{framed}
\usepackage{graphicx}
\usepackage{amsmath}
\usepackage{chngpage}
%\usepackage{bigints}

\usepackage{vmargin}
% left top textwidth textheight headheight
% headsep footheight footskip
\setmargins{2.0cm}{2.5cm}{16 cm}{22cm}{0.5cm}{0cm}{1cm}{1cm}
\renewcommand{\baselinestretch}{1.3}

\setcounter{MaxMatrixCols}{10}

\begin{document}
\large 


%%- http://webpages.iust.ac.ir/matashbar/teaching/schaum_probability.pdf

%%-- PAge 287
%%-- 9.13


A corporate computing center has two computers of the same capacity. The jobs arriving at the 
center are of two types, internal jobs and external jobs. These jobs have Poisson arrival times 
with rates 18 and 15 per hour, respectively. The service time for a job is an exponential r.v. with 
mean 3 minutes. 
\begin{enumerate}[(a)]
    \item Find the average waiting time per job when one computer is used exclusively for internal 
jobs and the other for external jobs. 
    \item Find the average waiting time per job when two computers handle both types of jobs. 
\end{enumerate}


\section*{Solution}

\subsection*{Part (a)}
\begin{itemize}
    \item (a) When the computers are used separately, we treat them as two M/M/1 queueing systems. 
    
        \item Let W,, and 
W,, be the average waiting time per internal job and per external job, respectively. 

    \item For internal jobs, 

A1 = = & and p1 = 3. Then, from Eq. (9.16), 
- 3 
10 Wq I = ---- = 27 min 4(4 - 6) 
    \item For external jobs, 1, = = $ and p2 = 5, and 
1 
;I Wq2 = ---- - 9 min +(3 - $1 
\end{itemize}


\subsection*{Part (b)}
\begin{itemize}
    \item (b) When two computers handle both types of jobs, we model the computing service as an M/M/2 
queueing system with 
\item Now, substituting s = 2 in Eqs. (9.20), (9.22), (9.24), and (9.25), we get 
\item Thus, from Eq. (9.54), the average waiting time per job when both computers handle both types of jobs 
is given by 
2(%) %= mu 11 = 6.39 min - (%)21 
\item From these results, we see that it is more efficient for both computers to handle both types of jobs.
\end{itemize}

\end{document}
