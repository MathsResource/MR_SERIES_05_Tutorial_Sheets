\documentclass[a4paper,12pt]{article}
%%%%%%%%%%%%%%%%%%%%%%%%%%%%%%%%%%%%%%%%%%%%%%%%%%%%%%%%%%%%%%%%%%%%%%%%%%%%%%%%%%%%%%%%%%%%%%%%%%%%%%%%%%%%%%%%%%%%%%%%%%%%%%%%%%%%%%%%%%%%%%%%%%%%%%%%%%%%%%%%%%%%%%%%%%%%%%%%%%%%%%%%%%%%%%%%%%%%%%%%%%%%%%%%%%%%%%%%%%%%%%%%%%%%%%%%%%%%%%%%%%%%%%%%%%%%
\usepackage{eurosym}
\usepackage{vmargin}
\usepackage{amsmath}
\usepackage{framed}
\usepackage{multicol}
\usepackage{graphics}
\usepackage{epsfig}
\usepackage{subfigure}
\usepackage{enumerate}
\usepackage{fancyhdr}

\setcounter{MaxMatrixCols}{10}
%TCIDATA{OutputFilter=LATEX.DLL}
%TCIDATA{Version=5.00.0.2570}
%TCIDATA{<META NAME="SaveForMode"CONTENT="1">}
%TCIDATA{LastRevised=Wednesday, February 23, 201113:24:34}
%TCIDATA{<META NAME="GraphicsSave" CONTENT="32">}
%TCIDATA{Language=American English}

\pagestyle{fancy}
\setmarginsrb{20mm}{0mm}{20mm}{25mm}{12mm}{11mm}{0mm}{11mm}
\lhead{MathsResource} \chead{Introduction to Calculus} \rhead{Series Summations} %\input{tcilatex}
\begin{document}


\begin{enumerate}
	
	\item  Express the following repeating decimal number as a simple fraction. Show your workings.
	
	\[0.2162162162162....\]
	%------------------------%
\item 
Compute the summations of the following infinite series
\begin{multicols}{2}
\begin{enumerate}[(a)]
\item $1 + 0.2 + 0.04 + \ldots$
\item $1 - 0.2 + 0.04 - \ldots$
\item $20 + 5 + 1.25 + \ldots$
\item $- 20 + 5 - 1.25 + \ldots$
\end{enumerate}
\end{multicols}
\item 
Find the value to which each of the following series converges.

\[\sum_{n=1}^{\infty} \frac{3}{4^{n-1}}\]
%- http://en.wikibooks.org/wiki/Calculus/Sequences_and_Series/Exercises


\item  Compute the following summation

\begin{multicols}{4}
	\begin{enumerate}[(a)]
\item	\[ \sum_{i=1}^{30} i \]
\item   \[ \sum_{i=1}^{65} i \]
\item	\[ \sum_{i=1}^{37} i \]
\item 	\[ \sum_{i=38}^{65} i \]
\item   \[ \sum_{i=10}^{50} i  \]
\item \[ \sum_{i=21}^{40} i \]
\item \[ \sum_{i=1}^{80} i \]
	\end{enumerate}
\end{multicols}

\item The second term $u_2$ of a geometric sequence is 24. The third term $u_3$ is -96. 
\begin{itemize}
\item[(a)] Find the common ratio $r$. 
\item[(b)] Find the first and fourth term $u_1$ and $u_4$.
\end{itemize}
	
\item Three consecutive terms of an arithmetic series are \[4x - 1, 2x +11, 3x + 41. \]
	Find the value of $x$.
	%answer is 460
	
\item Suppose that the following term is the general term for a series. Use the Ratio Test to test this series for convergence
	
	\[u_n=\frac{n!n!}{(2n)!}\]

\item 	Three consecutive terms of an arithmetic series are 
	\[4x+4,6x-6,7x-5\]
Find the values for $x$ and constant difference $d$.
	%answer is 460
\item Find the sum of the following geometric series: 
		\[5 + 15 + 45 +  \ldots + 3645\]
\item  Find the sum of the following geometric series: 
		\[2 + 6 + 18 +  \ldots + 39366		\]

\item Find the value to which each of the following series converges.

\[\sum_{n=1}^{\infty} \frac{3}{4^{n-1}}\]
\item What does the following sequence converge to?
 \[ u_n  = \frac{n+3}{4n+2}\]

\item Find the sum of the telescoping series  
\[ \sum^{\infty}_{n=1} \frac{2}{(2n-1)(2n+1)}\]




	\item  Compute the following summation
	
	\[ \sum_{i=1}^{80} i \]



	
	\item  Express the following repeating decimal number as a simple fraction. S
	]how your workings.
	
	\[0.243243243243243....\]


	%------------------------%
	%%updated for 2016

	\item 
	The second term $u_2$ of a geometric sequence is 24. The third term $u_3$ is -96. \\  Answer the following questions. Both questions are worth 2 Marks each.
	\begin{itemize}
		\item[(a)] Find the common ratio $r$. 
		\item[(b)] Find the first and fourth term $u_1$ and $u_4$.
	\end{itemize}
	
	\item 	Three consecutive terms of an arithmetic series are \[7x-22, 3x+2 , 5x-4. \]
	Find the value of $x$.
	%answer is 460
	
	\item 
	The second term $u_2$ of a geometric sequence is 24. The third term $u_3$ is -96. \\ \bigskip Answer the following questions. Both questions are worth 2 Marks each.
	\begin{itemize}
		\item[(a)] Find the common ratio $r$. 
		\item[(b)] Find the first and fourth term $u_1$ and $u_4$.
	\end{itemize}
 .




		
	\item Express the following repeating decimal number as a simple fraction. Show your workings.
	
	\[0.162162162162....\]


	\item 
	The second term $u_3$ of a geometric sequence is 24. The third term $u_4$ is -72. \\  Answer the following questions. Both questions are worth 2 Marks each.
	\begin{itemize}
		\item[(a)] Find the common ratio $r$. 
		\item[(b)] Find the first and fourth term $u_1$ and $u_5$.
	\end{itemize}
\item Find the sum of the arithmetic progression
{

\[ 11 + 13 + 15 + \dots + 49 + 51 \]
}

\item down $S_n$ and $S_{\infty}$ of the infinite geometric series.
\[ 0.7 + 0.07 + 0.007 + 0.0007 + \ldots  \]


\item Write down $S_n$ and $S_{\infty}$ of the infinite geometric series.
\[ 1 + \frac{3}{4} + \left( \frac{3}{4} \right)^2 + \left( \frac{3}{4} \right)^3 + \ldots  \]



\item Find $S_n$, the sum of $n$ terms, of the geometric series

\[  2 + \frac{2}{3} + \frac{2}{3^2} + \frac{2}{3^3} +  \ldots + \frac{2}{3^{n-1}} \]

If $S_n$ = 242/81, find the value of $n$.


\item Write down $S_n$ and $S_{\infty}$ of the infinite geometric series.
\[ 0.7 + 0.07 + 0.007 + 0.0007 + \ldots  \]


%----------------------------------------%


	%%updated for 2016
	\item  Suppose that the following term is the general term for a series. Use the Ratio Test to test this series for convergence
	
	\[u_n=\frac{(2n)!}{n!n!}\]

%- http://en.wikibooks.org/wiki/Calculus/Sequences_and_Series/Exercises	
\item  Suppose the general term $u_n$ is given as  $\displaystyle{u_n = \frac{2^n}{(2n)!} } .$ State $u_{n+1}$ and hence calculate a simplified expression for $r$, where 
	$\displaystyle{r = \frac{u_{n+1}}{u_n} }$

%%%%%%%%%%%%%%%%%%%%%%%%%%%%%%%%%%%%%%%%%%%%%%%%%%%%%%%%%%%
\item  Suppose the general term $u_n$ is given as $\displaystyle{u_n = \frac{n!}{5^n} }$ . State $u_{n+1}$ and hence calculate a simplified expression for $r$, where 
	{$\displaystyle{r = \frac{u_{n+1}}{u_n}}$	}
%%%%%%%%%%%%%%%%%%%%%%%%%%%%%%%%%%%%%%%%%%%%%%%%%%%%%%%%%%%
\item  Suppose the general term $u_n$ is given as  $\displaystyle{u_n = \frac{5^n}{n!} } .$ State $u_{n+1}$ and hence calculate a simplified expression for $r$, where 
	{	$\displaystyle{r = \frac{u_{n+1}}{u_n}}$	}
%%%%%%%%%%%%%%%%%%%%%%%%%%%%%%%%%%%%%%%%%%%%%%%%%%%%%%%%%%%
\item Suppose the general term $u_n$ is given as  $u_n = \displaystyle{\frac{3^n}{n!} }$ . State $u_{n+1}$ and hence calculate a simplified expression for $r$, where 
	{	$\displaystyle{r = \frac{u_{n+1}}{u_n}}$	}


\item A sequence is given by the recurrence relation

\[u_{n+1} = u_n + n \mbox{ and }u_1 = 0\]

\begin{itemize}

\item[(i)] Calculate $u_3$, $u_4$, and $u_5$. 

\item[(ii)] Use induction to prove the following 

\end{itemize}

\[u_n = \frac{n(n-1)}{2} \mbox{ for all n = 1} \]



\item Write the following in $\sigma$ notation

\[ 1 + 4 + 7 + 10 + \ldots + (3n - 2)\]

Evaluate this when n = 100. 


\end{enumerate}
\end{document}
