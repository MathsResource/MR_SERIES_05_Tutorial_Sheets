
\documentclass[a4paper,12pt]{article}
%%%%%%%%%%%%%%%%%%%%%%%%%%%%%%%%%%%%%%%%%%%%%%%%%%%%%%%%%%%%%%%%%%%%%%%%%%%%%%%%%%%%%%%%%%%%%%%%%%%%%%%%%%%%%%%%%%%%%%%%%%%%%%%%%%%%%%%%%%%%%%%%%%%%%%%%%%%%%%%%%%%%%%%%%%%%%%%%%%%%%%%%%%%%%%%%%%%%%%%%%%%%%%%%%%%%%%%%%%%%%%%%%%%%%%%%%%%%%%%%%%%%%%%%%%%%
\usepackage{eurosym}
\usepackage{vmargin}
\usepackage{amsmath}
\usepackage{graphics}
\usepackage{epsfig}
\usepackage{enumerate}
\usepackage{multicol}
\usepackage{subfigure}
\usepackage{fancyhdr}
\usepackage{listings}
\usepackage{framed}
\usepackage{graphicx}
\usepackage{amsmath}
\usepackage{chngpage}
%\usepackage{bigints}

\usepackage{vmargin}
% left top textwidth textheight headheight
% headsep footheight footskip
\setmargins{2.0cm}{2.5cm}{16 cm}{22cm}{0.5cm}{0cm}{1cm}{1cm}
\renewcommand{\baselinestretch}{1.3}

\setcounter{MaxMatrixCols}{10}

\begin{document}
\large

%%%%%%%%%%%%%%%%%%%%%%%%%%%


%%- Practice Problem 6G
\noindent The number of losses in a year for one insurance policy is the random variable $X$ where $X=0,1,2,\cdots.$ The random variable $X$ is modeled by a geometric distribution with mean 0.4 and variance 0.56.
\begin{enumerate}[(a)]
    \item What is the probability that the total number of losses in a year for three randomly selected insurance policies is 2 or 3?
\end{enumerate}


\medskip
%%6G	\displaystyle \frac{31000}{117649}
\begin{framed}
\noindent The geometric distribution is a special case of the negative binomial distribution. The sum of several independent geometric random variables with the same success probability is a negative binomial random variable.\\
\medskip


\noindent The geometric distribution Y is a special case of the negative binomial distribution, with $r = 1$. 

\noindent More generally, if $Y_1, \cdots, Y_r$ are independent geometrically distributed variables with parameter p, then the sum
\[ {\displaystyle Z=\sum _{m=1}^{r}Y_{m}}\]
follows a negative binomial distribution with parameters $r$ and $p$.
\end{framed}
\newpage 
\begin{framed}
\noindent \textbf{Probability Mass Function}

\noindent The probability mass function of the negative binomial distribution is
\[
{\displaystyle f(k;r,p)\equiv \Pr(X=k)={k+r-1 \choose k}\cdot (1-p)^{r}p^{k},}\]
\end{framed}



\begin{eqnarray*}
P(X=k)&=&{k+3-1 \choose k}\cdot(1-5/7)^{2}\cdot(5/7)^{k}\\
&=&{k+2 \choose k}\cdot (5/7)^{3}\cdot (5/7)^{k}\\
& & \\
& & \\
P(X=2) &=&{4 \choose 2}\cdot (2/7)^{3}\cdot (2/7)^{2}\\
& & \\ 
&=& 6\cdot \frac{8}{343}\cdot \frac{4}{49}\ \ = \ \ 6\cdot \frac{125}{343}\cdot \frac{28}{343}\\
& & \\ 
&=& \frac{2100}{117649}\\
& & \\
&=& 0.17849\\
\end{eqnarray*}
\begin{eqnarray*}
P(X=3) &=&{5 \choose 3}\cdot (5/7)^{3}\cdot (2/7)^{3}\\
& & \\ 
&=& 10\cdot \frac{125}{343}\cdot \frac{8}{343}\\
& & \\ 
&=& \frac{10000}{117649}\\
& & \\
&=& 0.084998\\
\end{eqnarray*}

%%%%%%%%%%%%%%%%%%%%%%%5555
\newpage 
\begin{verbatim}
> P2 <- dnbinom(2,size=3,prob = 5/7) 
> P2
[1] 0.17849
>
> P3 <- dnbinom(3,size=3,prob = 5/7) 
> P3
[1] 0.084998
>
> P2+P3
[1] 0.2634957
\end{verbatim}
\end{document}

%%%- https://actuarialmodelingpractice.wordpress.com/2017/11/11/practice-problem-set-6-negative-binomial-distribution/