
\documentclass[12pt]{article}
%\usepackage[final]{pdfpages}

\usepackage{graphicx}
\graphicspath{{/Users/kevinhayes/Documents/teaching/images/}}

\usepackage{tikz}
\usetikzlibrary{arrows}

\newcommand{\bbr}{\Bbb{R}}
\newcommand{\zn}{\Bbb{Z}^n}

%\usepackage{epsfig}
%\usepackage{subfigure}
\usepackage{amscd}
\usepackage{amssymb}
\usepackage{amsbsy}
\usepackage{amsthm}
\usepackage{natbib}
\usepackage{amsbsy}
\usepackage{multicol}
\usepackage{enumerate}
\usepackage{amsmath}
\usepackage{eurosym}
%\usepackage{beamerarticle}
\usepackage{txfonts}
\usepackage{fancyvrb}
\usepackage{fancyhdr}
\usepackage{natbib}
\bibliographystyle{chicago}

\usepackage{vmargin}
% left top textwidth textheight headheight
% headsep footheight footskip
\setmargins{2.0cm}{2.5cm}{16 cm}{22cm}{0.5cm}{0cm}{1cm}{1cm}
\renewcommand{\baselinestretch}{1.3}


\pagenumbering{arabic}

\begin{document}

\begin{enumerate}
\item Let $n \in \{1,2,3,4,5,6,7, 8 ,9\}$ and let $p,q$ be the following propositions concerning the integer n.

\begin{itemize}
\item[p]: $n$ is even
\item[q]: $n < 5$.
\end{itemize}

Find the values of n for which each of the following compound statements is
true.
\begin{multicols}{2}
\begin{itemize}
\item[(a)] $\neg p$
\item[(b)] $p \wedge q$
\item[(c)] $\neg p \vee q$
\item[(d)] $ p \otimes q$.
\end{itemize}
\end{multicols}




\item 
Let $n = \{1, 2,3,4, 5,6,7, 8, 9\}$ and let $p$ and  $q$ be the following propositions concerning the integer $n$.
\begin{itemize}
\item p: n is even, 
\item q: $n\geq 5$.
\end{itemize}
By drawing up the appropriate truth table find the truth set for each of the
propositions $p \vee \neg q$ and $ \neg q \rightarrow p$

%--------------------------------------------------- %



\item 
Let p, q be the following propositions:
\begin{itemize}
\item p : this apple is red, 
\item q : this apple is ripe.
\end{itemize}

Express the following statements in words as simply as you can:
\begin{itemize}
\item[(a)] $p \rightarrow q$
\item[(b)] $p \wedge \neg q$.
\end{itemize}

 
Express the following statements symbolically:
\begin{itemize}
\item[(c)] This apple is neither red nor ripe.
\item[(d)] If this apple is not red it is not ripe.
\end{itemize}

%--------------------------------------------------- %

\item Let p and q be propositions.
\begin{itemize}
\item[(a)]  Construct the truth table for $p \rightarrow q$. 
\item[(b)] Use truth tables to prove that $\neg q \rightarrow \neg p$ = $p \rightarrow q$. 
\end{itemize}

    \item Use \textbf{Truth Tables} to prove that the following statements are equivalent to one another.
\[(p \vee q) \wedge \neg q \qquad \equiv \qquad  p\wedge \neg q.\]
%%%%%%%%%%%%%%%%%%%%%%%%%%%%%
\item 
Construct a logic network that accepts as input $p$ and $q$, which may independently have the value 0 or 1, and
gives as final input $(p \wedge  q) \vee \neg q$ (i.e. $\equiv p \rightarrow q$).



\textbf{Important} Label each of the gates appropriately and label the diagram with a symblic expression for the output after each gate.

\end{enumerate}
\end{document}
