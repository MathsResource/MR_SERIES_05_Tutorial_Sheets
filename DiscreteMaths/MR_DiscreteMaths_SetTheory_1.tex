\documentclass[a4paper,12pt]{article}

%%%%%%%%%%%%%%%%%%%%%%%%%%%%%%%%%%%%%%%%%%%%%%%%%%%%%%%%%%%%%%%%%%%%%%%%%%%%%%%%%%%%%%%%%%%%%%%%%%%%%%%%%%%%%%%%%%%%%%%%%%%%%%%%%%%%%%%%%%%%%%%%%%%

\usepackage{eurosym}
\usepackage{vmargin}
\usepackage{amsmath}
\usepackage{graphics}
\usepackage{epsfig}
\usepackage{enumerate}
\usepackage{multicol}
\usepackage{subfigure}
\usepackage{fancyhdr}
\usepackage{listings}
\usepackage{framed}
\usepackage{graphicx}
\usepackage{amsmath}
\usepackage{chngpage}

%\usepackage{bigints}



\usepackage{vmargin}

% left top textwidth textheight headheight

% headsep footheight footskip

\setmargins{2.0cm}{2.5cm}{16 cm}{22cm}{0.5cm}{0cm}{1cm}{1cm}

\renewcommand{\baselinestretch}{1.3}

\setcounter{MaxMatrixCols}{10}



\begin{document}

\section*{Set Theory Tutorial 1}
\begin{enumerate}
    \item Describe the following set by the rules of inclusion method.
    \begin{enumerate}[(a)]
        \item $\{12,13,14,15,16,17\}$
\item $\{0,5,-5,10,-10,15,-15,.....\}$
\item $\{6,8,10,12,14,16,18\}$
    \end{enumerate}
%=======================+============================================================== %



\item Describe the following set by the listing method the set of positive multiples of 3 which are less than 20.
\item Describe the following set by the listing method
\[ \{ 2r+1 : r \in Z^{+} and r \leq 5  \} \]

\item For each of the following sets, write out the set using the listing method.
Also write down the cardinality of each set.

\begin{itemize}
\item[(a)] $\{ 10^m : -2 \leq m \leq 4, m \in \mathbb{Z} \} $
\item[(b)]  $ \{ \frac{1}{n}: 1 < n < 4, n \in \mathbb{Z} \} $
\end{itemize}

\item For each of the following sets, write out the set using the listing method.
Also write down the cardinality of each set.

\begin{enumerate}[(a)] 
\item $\{ s : s $ is an negative integer $ -10 \leq s \leq 0 \}$
\item $\{ t : t $ is an even number $ 1 \leq t \leq 20 \}$
\item $\{ u : u $ is a prime number $ 1 \leq u \leq 20 \}$
\item $\{ v : v $ is a multiple of 3 $ 1 \leq v \leq 20 \}$
\end{enumerate}
\item Let $A = \{2n+1 : n \in Z^{+}\}$ be a set of numbers. Describe the set $A$ by the listing method.

\item 
U = {natural numbers}; $A = \{2, 4, 6, 8, 10\}$; $B = \{1, 3, 6, 7, 8\}$. State whether each of the following is true or false:
\begin{itemize}
\item[(a)] $A \subset U$
\item[(b)] $B \subseteq A$
\item[(c)] $\emptyset \subset U$
\end{itemize}

\item 
For each of the following sets, write out the set using the listing method.
Also write down the cardinality of each set.
\begin{enumerate}[(a)]
	\item $\{ s :  \mbox{ s is an odd integer and } 2 \leq s \leq 10 \}$
	\item $\{ 2m :  m \in Z \mbox{ and }5 \leq m \leq 10 \}$
	\item $\{ 2^t :  t \in Z \mbox{ and } 0 \leq t \leq 5 \}$
\end{enumerate}



\item Let $A$ and $B$ be subsets of universal set $U$. Use the membership rule to prove that
\[(A^\prime  \cap B)^\prime = A \cup B^\prime\]
Shade the region corresponding to this set on a Venn Diagram

\item Given the universal set $$\mathcal{U} = {1,2,3,4,5,6,7,8,9}$$ and the subsets $A=\{1,3,5,7\}$
$B = \{6,7,8,9\}$ list the set $A^\prime \cap B)^\prime$
\item Shade in the following areas on Venn diagrams.
\begin{multicols}{3}
\begin{itemize}
\item[(a)] $A^\prime\; \cup\; B$

\item[(b)] $A \cap\; B^\prime\;$

\item[(c)] $(A \cap\; B)^\prime\;$

\item[(d)] $A^\prime\; \cup\; B^\prime\;$

\item[(e)] $(A \cup\; B)^\prime\;$

\item[(f)] $A^\prime\; \cap\; B^\prime\;$

\end{itemize}
\end{multicols}
%-------------------------------%
\item 
The following sets have been defined using the \textbf{Building Method} of notation. Re-write them by listing \textbf{some} of the elements.
\begin{enumerate}
\item $\{p | p$ is a capital city, p is in Europe$\}$
\item $\{x | x = 2n - 5,$ x and n are natural numbers$\}$
\item $\{y | 2y^2 = 50,$ y is an integer$\}$
\item $\{z | z = n^3,$ z and n are natural numbers$\}$
\end{enumerate}


\item 
Consider the universal set $U$ such that
\[U=\{1,2,3,4,5,6,7,8,9\} \]
and the sets
\[A=\{2,5,7,9\} \]
\[B=\{2,4,6,8,9\} \]
\begin{multicols}{2}
\begin{itemize}
	\item[(a)] $A-B$
	\item[(b)] $A \otimes B$
	\item[(c)] $A \cap B$
	\item[(d)] $A \cup B$
	\item[(e)] $A^{c} \cap B^{c}$
	\item[(f)] $A^{c} \cup B^{c}$
\end{itemize}
\end{multicols}


\item Let A, B be subsets of the universal set $\mathcal{U}$.

Use membership tables to prove De Morgan's Laws.

\end{enumerate}



\end{document}
