%% MR_LinearAlgebra_Tutorial_5_Planes.pdf
%% https://www.amazon.com/clouddrive/share/i8yTqHVhtZw7cQPOKAqtQ6Jj0n0rMCtqJ1NRz0cOURj


\documentclass[12pt, a4paper]{report}

\usepackage{epsfig}
\usepackage{subfigure}
%\usepackage{amscd}
\usepackage{amssymb}
\usepackage{graphicx}
%\usepackage{amscd}
\usepackage{amssymb}
\usepackage{enumerate}
\usepackage{subfiles}
\usepackage{framed}
\usepackage{subfiles}
\usepackage{amsthm, amsmath}
\usepackage{amsbsy}
\usepackage{framed}
\usepackage[usenames]{color}
\usepackage{listings}
\lstset{% general command to set parameter(s)
basicstyle=\small, % print whole listing small
keywordstyle=\color{red}\itshape,
% underlined bold black keywords
commentstyle=\color{blue}, % white comments
stringstyle=\ttfamily, % typewriter type for strings
showstringspaces=false,
numbers=left, numberstyle=\tiny, stepnumber=1, numbersep=5pt, %
frame=shadowbox,
rulesepcolor=\color{black},
,columns=fullflexible
} %
%\usepackage[dvips]{graphicx}
\usepackage{natbib}
\bibliographystyle{chicago}
\usepackage{vmargin}

\renewcommand{\baselinestretch}{1.5}
\pagenumbering{arabic}
\theoremstyle{plain}
\newtheorem{theorem}{Theorem}[section]
\newtheorem{corollary}[theorem]{Corollary}
\newtheorem{ill}[theorem]{Example}
\newtheorem{lemma}[theorem]{Lemma}
\newtheorem{proposition}[theorem]{Proposition}
\newtheorem{conjecture}[theorem]{Conjecture}
\newtheorem{axiom}{Axiom}
\theoremstyle{definition}
\newtheorem{definition}{Definition}[section]
\newtheorem{notation}{Notation}
\theoremstyle{remark}
\newtheorem{remark}{Remark}[section]
\newtheorem{example}{Example}[section]
\renewcommand{\thenotation}{}
\renewcommand{\thetable}{\thesection.\arabic{table}}
\renewcommand{\thefigure}{\thesection.\arabic{figure}}
\title{Research notes: linear mixed effects models}
\author{ } \date{ }


\begin{document}
\section*{Linear Algebra Tutorial Sheet : Lines and Planes}
\begin{enumerate}
    \item 


%2011 Question 2
\begin{itemize}
\item[(i)]  Give the general form of the equation of the plant $\pi$ in $\mathbb{R}^3$ passing throughthe point $P_0 =(1,0,2)$ with the vector $n=(-5,3,2)$ as the normal.


\item[(ii)]  Show that the point $Q=(1,-1,1)$ does not lie in the plane $\pi$ and find its distance from $\pi$.
\end{itemize}

\item
%2011 Question 2
\begin{itemize}
\item[(i)]  Give the general form of the equation of the plant $\pi$ in $\mathbb{R}^3$ passing through the point $P_0 =(1,0,2)$ with the vector $n=(-5,5,2)$ as the normal.


\item[(ii)]  Show that the point $Q=(1,-1,1)$ does not lie in the plane $\pi$ and find its distance from $\pi$.
\end{itemize}

%\subsection*{Part D. Planes (5Marks)}
%\begin{enumerate}
%\item Find the general form of the equation of the plane $\pi$ in $\mathbb{R}^3$ which passes through the point 
%$P=(3,1,6)$ and is orthogonal to the vector $n=(1,7,-2)$. %\marks{3}
%
%\item Show that the point $Q=(1,-1,1)$ does not lie in the plane $\pi$ and find its distance from $\pi$. %\marks{ 2}
%\end{enumerate}



%\subsection*{Part D. Planes (5Marks)}
%\begin{enumerate}
%\item Find the general form of the equation of the plane $\pi$ in $\mathbb{R}^3$ which passes through the point 
%$P=(3,1,6)$ and is orthogonal to the vector $n=(1,7,-2)$. %\marks{3}
%
%\item Show that the point $Q=(1,-1,1)$ does not lie in the plane $\pi$ and find its distance from $\pi$. %\marks{ 2}
%\end{enumerate}
%=========================================================%

\item \begin{enumerate}
\item Find the general form of the equation of the plane $\pi$ in $\mathbb{R}^3$ which passes through the point 
$P=(3,1,6)$ and is orthogonal to the vector $n=(1,7,-2)$. %\marks{3}

\item Show that the point $Q=(1,-1,1)$ does not lie in the plane $\pi$ and find its distance from $\pi$. 
\end{enumerate}
%=========================================================%
\end{enumerate}
\end{document}
