\documentclass[a4paper,12pt]{article}
%%%%%%%%%%%%%%%%%%%%%%%%%%%%%%%%%%%%%%%%%%%%%%%%%%%%%%%%%%%%%%%%%%%%%%%%%%%%%%%%%%%%%%%%%%%%%%%%%%%%%%%%%%%%%%%%%%%%%%%%%%%%%%%%%%%%%%%%%%%%%%%%%%%%%%%%%%%%%%%%%%%%%%%%%%%%%%%%%%%%%%%%%%%%%%%%%%%%%%%%%%%%%%%%%%%%%%%%%%%%%%%%%%%%%%%%%%%%%%%%%%%%%%%%%%%%
\usepackage{eurosym}
\usepackage{vmargin}
\usepackage{amsmath}
\usepackage{graphics}
\usepackage{epsfig}
\usepackage{enumerate}
\usepackage{multicol}
\usepackage{subfigure}
\usepackage{fancyhdr}
\usepackage{listings}
\usepackage{framed}
\usepackage{graphicx}
\usepackage{amsmath}
\usepackage{chngpage}
%\usepackage{bigints}

\usepackage{vmargin}
% left top textwidth textheight headheight
% headsep footheight footskip
\setmargins{2.0cm}{2.5cm}{16 cm}{22cm}{0.5cm}{0cm}{1cm}{1cm}
\renewcommand{\baselinestretch}{1.3}

\setcounter{MaxMatrixCols}{10}
\begin{document}


\noindent The following table defines the various activities in a small project: 
\begin{center}
\begin{tabular}{|c|c|} \hline
Activity & Completion time (weeks) \\ \hline
A & 5 \\ \hline
B & 7 \\ \hline
C & 3 \\ \hline
D & 5 \\ \hline
E & 5 \\ \hline
F & 3 \\ \hline
G & 9 \\ \hline
H & 4 \\ \hline
I & 9 \\ \hline
J & 1 \\ \hline
\end{tabular}
\end{center}
\noindent The immediate precedence relationships are: Activity 
\begin{center}
\begin{tabular}{|c|c|}\hline
Activity & Depends on  \\
         & completion of \\
         \hline
H & A \\ \hline
D & J \\ \hline
B,C & I \\ \hline
E,J,I,A,F & G \\ \hline
\end{tabular}
\end{center}
(i.e. Activity A must be finished before H can start etc).\\


\noindent In addition there must be a time lag of at least 4 weeks between the end of activity A and the start of activity D. 
\begin{enumerate}[(a)]
    \item Draw the network diagram. 
    \item Calculate the critical activities, the overall project completion time and the float times for each activity. 
    \item Explain clearly (with reasons) the effect upon the overall project completion time if the completion time for activity A increases from 5 weeks to 6 weeks and the time lag between the end of activity A and the start of activity D also increases from 4 weeks to 6 weeks. 
\item  Suppose now that, excluding (c) above, the project manager wishes to reduce the completion time calculated at (b) above by one week and all activities in the project can be crashed if necessary at a cost of £200 for each week crashed. The project manager wishes to choose just one activity to crash. 
\end{enumerate}

\begin{enumerate}[(i)]
    \item Which activities should not be considered for crashing and why? 
    \item Which activities should be considered for crashing and why? 
    \item Which activity would you recommend be crashed and what would be the effect of your recommendation?
\end{enumerate} 
\end{document}
