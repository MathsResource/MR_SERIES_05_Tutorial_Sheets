\documentclass[a4paper,12pt]{article}
%%%%%%%%%%%%%%%%%%%%%%%%%%%%%%%%%%%%%%%%%%%%%%%%%%%%%%%%%%%%%%%%%%%%%%%%%%%%%%%%%%%%%%%%%%%%%%%%%%%%%%%%%%%%%%%%%%%%%%%%%%%%%%%%%%%%%%%%%%%%%%%%%%%%%%%%%%%%%%%%%%%%%%%%%%%%%%%%%%%%%%%%%%%%%%%%%%%%%%%%%%%%%%%%%%%%%%%%%%%%%%%%%%%%%%%%%%%%%%%%%%%%%%%%%%%%
\usepackage{eurosym}
\usepackage{vmargin}
\usepackage{amsmath}
\usepackage{graphics}
\usepackage{epsfig}
\usepackage{enumerate}
\usepackage{multicol}
\usepackage{subfigure}
\usepackage{fancyhdr}
\usepackage{listings}
\usepackage{framed}
\usepackage{graphicx}
\usepackage{amsmath}
\usepackage{chngpage}
%\usepackage{bigints}

\usepackage{vmargin}
% left top textwidth textheight headheight
% headsep footheight footskip
\setmargins{2.0cm}{2.5cm}{16 cm}{22cm}{0.5cm}{0cm}{1cm}{1cm}
\renewcommand{\baselinestretch}{1.3}

\setcounter{MaxMatrixCols}{10}

\begin{document}
\large



%%- https://online.stat.psu.edu/stat414/lesson/26/26.1

  
        Example 26-2

History suggests that scores on the Math portion of the Standard Achievement Test (SAT) are normally distributed with a mean of 529 and a variance of 5732. 

History also suggests that scores on the Verbal portion of the SAT are normally distributed with a mean of 474 and a variance of 6368. 

Select two students at random. Let \(X\) denote the first student's Math score, and let \(Y\) denote the second student's Verbal score. What is \(P(X&gt;Y)\)?


\subsection*{Solution}

We can find the requested probability by noting that \(P(X&gt;Y)=P(X-Y&gt;0)\), and then taking advantage of what we know about the distribution of \(X-Y\).

That is, \(X-Y\) is normally distributed with a mean of 55 and variance of 12100 as the following calculation illustrates:

<p class="text-align-center">\((X-Y)\sim N(529-474,(1)^2(5732)+(-1)^2(6368))=N(55,12100)\)

Then, finding the probability that \(X\) is greater than \(Y\) reduces to a normal probability calculation:

<p class="text-align-center">\begin{align} P(X&gt;Y) &amp;=P(X-Y&gt;0)\\ &amp;= P\left(Z&gt;\dfrac{0-55}{\sqrt{12100}}\right)\\ &amp;= P\left(Z&gt;-\dfrac{1}{2}\right)=P\left(Z&lt;\dfrac{1}{2}\right)=0.6915\\ \end{align}

That is, the probability that the first student's Math score is greater than the second student's Verbal score is 0.6915.


 